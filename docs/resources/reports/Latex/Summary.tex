\documentclass[a4paper,landscape,10pt]{article}
\usepackage[paper=a4paper,landscape,left=20mm,right=20mm,top=20mm,bottom=20mm]{geometry}
\usepackage{longtable}
\usepackage{fancyhdr}
\usepackage[pdftex]{color}
\usepackage{colortbl}
\definecolor{green}{rgb}{0.04,0.68,0.04}
\definecolor{orange}{rgb}{0.97,0.65,0.12}
\definecolor{red}{rgb}{0.75,0.04,0.04}
\definecolor{gray}{rgb}{0.86,0.86,0.86}

\usepackage[pdftex,
            colorlinks=true, linkcolor=red, urlcolor=green, citecolor=red,%
            raiselinks=true,%
            bookmarks=true,%
            bookmarksopenlevel=1,%
            bookmarksopen=true,%
            bookmarksnumbered=true,%
            hyperindex=true,% 
            plainpages=false,% correct hyperlinks
            pdfpagelabels=true%,% view TeX pagenumber in PDF reader
            %pdfborder={0 0 0.5}
            ]{hyperref}

\hypersetup{pdftitle={Coverage Report},
            pdfauthor={ReportGenerator - 5.1.14.0}
           }

\pagestyle{fancy}
\fancyhead[LE,LO]{\leftmark}
\fancyhead[R]{\thepage}
\fancyfoot[C]{ReportGenerator - 5.1.14.0}

\begin{document}

\setcounter{secnumdepth}{-1}
\section{Summary}
\begin{longtable}[l]{ll}
\textbf{Generated on:} & 18.01.2023 - 20:03:56\\
\textbf{Parser:} & OpenCover\\
\textbf{Assemblies:} & 1\\
\textbf{Classes:} & 4\\
\textbf{Files:} & 5\\
\textbf{Covered lines:} & 83\\
\textbf{Uncovered lines:} & 42\\
\textbf{Coverable lines:} & 125\\
\textbf{Total lines:} & 260\\
\textbf{Line coverage:} & 66.4\% (83 of 125)\\
\textbf{Covered branches:} & 3\\
\textbf{Total branches:} & 6\\
\textbf{Branch coverage:} & 50\% (3 of 6)\\
\textbf{Covered methods:} & 14\\
\textbf{Total methods:} & 24\\
\textbf{Method coverage:} & 58.3\% (14 of 24)\\
\end{longtable}
\section{Risk Hotspots}
No risk hotspots found.
\section{Coverage}
\begin{longtable}[l]{|l|r|r|r|r|r|r|r|}
\hline
\textbf{Name} & \textbf{Covered} & \textbf{Uncovered} & \textbf{Coverable} & \textbf{Total} & \textbf{Line coverage} & \textbf{Branch coverage} & \textbf{Method coverage}\\
\hline
\textbf{Sample} & \textbf{83} & \textbf{42} & \textbf{125} & \textbf{260} & \textbf{66.4\%} & \textbf{50\%} & \textbf{58.3\%}\\
\hline
Sample.PartialClass & 12 & 10 & 22 & 53 & 54.5\% & 50\% & 50\%\\
\hline
Sample.TestClass & 12 & 9 & 21 & 38 & 57.1\% & 50\% & 50\%\\
\hline
Test.Program & 35 & 9 & 44 & 84 & 79.5\% &  & 66.6\%\\
\hline
Test.TestClass2 & 24 & 14 & 38 & 85 & 63.1\% & 50\% & 60\%\\
\hline
\end{longtable}
\newpage
\section{Sample.PartialClass}
\subsection{Summary}
\begin{longtable}[l]{ll}
\textbf{Class:} & Sample.PartialClass\\
\textbf{Assembly:} & Sample\\
\textbf{File(s):} & \begin{minipage}[t]{12cm}{C:\textbackslash temp\textbackslash PartialClass.cs\\C:\textbackslash temp\textbackslash PartialClass2.cs}\end{minipage} \\
\textbf{Covered lines:} & 12\\
\textbf{Uncovered lines:} & 10\\
\textbf{Coverable lines:} & 22\\
\textbf{Total lines:} & 53\\
\textbf{Line coverage:} & 54.5\% (12 of 22)\\
\textbf{Covered branches:} & 1\\
\textbf{Total branches:} & 2\\
\textbf{Branch coverage:} & 50\% (1 of 2)\\
\textbf{Covered methods:} & 3\\
\textbf{Total methods:} & 6\\
\textbf{Method coverage:} & 50\% (3 of 6)\\
\end{longtable}
\subsection{Metrics}
\begin{longtable}[l]{|l|r|r|r|r|r|}
\hline
\textbf{Method} & \textbf{Branch coverage} & \textbf{Crap Score} & \textbf{Cyclomatic complexity} & \textbf{NPath complexity} & \textbf{Sequence coverage}\\
\hline
\textbf{ExecutedMethod\_1()} & 100\% & 1 & 1 & 0 & 100\%\\
\hline
\textbf{ExecutedMethod\_2()} & 100\% & 1 & 1 & 0 & 100\%\\
\hline
\textbf{UnExecutedMethod\_1()} & 0\% & 2 & 1 & 0 & 0\%\\
\hline
\textbf{UnExecutedMethod\_2()} & 0\% & 2 & 1 & 0 & 0\%\\
\hline
\end{longtable}
\subsection{File(s)}
\subsubsection{C:\textbackslash temp\textbackslash PartialClass.cs}
\begin{longtable}[l]{lrrll}
\textbf{} & \textbf{\#} & \textbf{Line} & \textbf{} & \textbf{Line coverage}\\
\cellcolor{gray} &  & \verb~1~ & & \verb~using System;~\\
\cellcolor{gray} &  & \verb~2~ & & \verb~~\\
\cellcolor{gray} &  & \verb~3~ & & \verb~namespace Test~\\
\cellcolor{gray} &  & \verb~4~ & & \verb~{~\\
\cellcolor{gray} &  & \verb~5~ & & \verb~    partial class PartialClass~\\
\cellcolor{gray} &  & \verb~6~ & & \verb~    {~\\
\cellcolor{gray} &  & \verb~7~ & & \verb~        public void ExecutedMethod_1()~\\
\cellcolor{green} & 1 & \verb~8~ & & \verb~        {~\\
\cellcolor{green} & 1 & \verb~9~ & & \verb~            Console.WriteLine("Test");~\\
\cellcolor{green} & 1 & \verb~10~ & & \verb~        }~\\
\cellcolor{gray} &  & \verb~11~ & & \verb~~\\
\cellcolor{gray} &  & \verb~12~ & & \verb~        public void UnExecutedMethod_1()~\\
\cellcolor{red} & 0 & \verb~13~ & & \verb~        {~\\
\cellcolor{red} & 0 & \verb~14~ & & \verb~            Console.WriteLine("Test");~\\
\cellcolor{red} & 0 & \verb~15~ & & \verb~        }~\\
\cellcolor{gray} &  & \verb~16~ & & \verb~~\\
\cellcolor{gray} &  & \verb~17~ & & \verb~        private int someProperty;~\\
\cellcolor{gray} &  & \verb~18~ & & \verb~~\\
\cellcolor{gray} &  & \verb~19~ & & \verb~        public int SomeProperty~\\
\cellcolor{gray} &  & \verb~20~ & & \verb~        {~\\
\cellcolor{red} & 0 & \verb~21~ & & \verb~            get { return this.someProperty; }~\\
\cellcolor{gray} &  & \verb~22~ & & \verb~~\\
\cellcolor{gray} &  & \verb~23~ & & \verb~            set~\\
\cellcolor{green} & 1 & \verb~24~ & & \verb~            {~\\
\cellcolor{orange} & 1 & \verb~25~ & & \verb~                if (value < 0)~\\
\cellcolor{green} & 1 & \verb~26~ & & \verb~                {~\\
\cellcolor{green} & 1 & \verb~27~ & & \verb~                    this.someProperty = 0;~\\
\cellcolor{green} & 1 & \verb~28~ & & \verb~                }~\\
\cellcolor{gray} &  & \verb~29~ & & \verb~                else~\\
\cellcolor{red} & 0 & \verb~30~ & & \verb~                {~\\
\cellcolor{red} & 0 & \verb~31~ & & \verb~                    this.someProperty = value;~\\
\cellcolor{red} & 0 & \verb~32~ & & \verb~                }~\\
\cellcolor{green} & 1 & \verb~33~ & & \verb~            }~\\
\cellcolor{gray} &  & \verb~34~ & & \verb~        }~\\
\cellcolor{gray} &  & \verb~35~ & & \verb~    }~\\
\cellcolor{gray} &  & \verb~36~ & & \verb~}~\\
\end{longtable}
\subsubsection{C:\textbackslash temp\textbackslash PartialClass2.cs}
\begin{longtable}[l]{lrrll}
\textbf{} & \textbf{\#} & \textbf{Line} & \textbf{} & \textbf{Line coverage}\\
\cellcolor{gray} &  & \verb~1~ & & \verb~using System;~\\
\cellcolor{gray} &  & \verb~2~ & & \verb~~\\
\cellcolor{gray} &  & \verb~3~ & & \verb~namespace Test~\\
\cellcolor{gray} &  & \verb~4~ & & \verb~{~\\
\cellcolor{gray} &  & \verb~5~ & & \verb~    partial class PartialClass~\\
\cellcolor{gray} &  & \verb~6~ & & \verb~    {~\\
\cellcolor{gray} &  & \verb~7~ & & \verb~        public void ExecutedMethod_2()~\\
\cellcolor{green} & 1 & \verb~8~ & & \verb~        {~\\
\cellcolor{green} & 1 & \verb~9~ & & \verb~            Console.WriteLine("Test");~\\
\cellcolor{green} & 1 & \verb~10~ & & \verb~        }~\\
\cellcolor{gray} &  & \verb~11~ & & \verb~~\\
\cellcolor{gray} &  & \verb~12~ & & \verb~        public void UnExecutedMethod_2()~\\
\cellcolor{red} & 0 & \verb~13~ & & \verb~        {~\\
\cellcolor{red} & 0 & \verb~14~ & & \verb~            Console.WriteLine("Test");~\\
\cellcolor{red} & 0 & \verb~15~ & & \verb~        }~\\
\cellcolor{gray} &  & \verb~16~ & & \verb~    }~\\
\cellcolor{gray} &  & \verb~17~ & & \verb~}~\\
\end{longtable}
\newpage
\section{Sample.TestClass}
\subsection{Summary}
\begin{longtable}[l]{ll}
\textbf{Class:} & Sample.TestClass\\
\textbf{Assembly:} & Sample\\
\textbf{File(s):} & \begin{minipage}[t]{12cm}{C:\textbackslash temp\textbackslash TestClass.cs}\end{minipage} \\
\textbf{Covered lines:} & 12\\
\textbf{Uncovered lines:} & 9\\
\textbf{Coverable lines:} & 21\\
\textbf{Total lines:} & 38\\
\textbf{Line coverage:} & 57.1\% (12 of 21)\\
\textbf{Covered branches:} & 1\\
\textbf{Total branches:} & 2\\
\textbf{Branch coverage:} & 50\% (1 of 2)\\
\textbf{Covered methods:} & 1\\
\textbf{Total methods:} & 2\\
\textbf{Method coverage:} & 50\% (1 of 2)\\
\end{longtable}
\subsection{Metrics}
\begin{longtable}[l]{|l|r|r|r|r|r|}
\hline
\textbf{Method} & \textbf{Branch coverage} & \textbf{Crap Score} & \textbf{Cyclomatic complexity} & \textbf{NPath complexity} & \textbf{Sequence coverage}\\
\hline
\textbf{SampleFunction()} & 66.67\% & 3.14 & 3 & 2 & 75\%\\
\hline
\textbf{SampleFunction()} & 0\% & 2 & 1 & 0 & 0\%\\
\hline
\end{longtable}
\subsection{File(s)}
\subsubsection{C:\textbackslash temp\textbackslash TestClass.cs}
\begin{longtable}[l]{lrrll}
\textbf{} & \textbf{\#} & \textbf{Line} & \textbf{} & \textbf{Line coverage}\\
\cellcolor{gray} &  & \verb~1~ & & \verb~using System;~\\
\cellcolor{gray} &  & \verb~2~ & & \verb~~\\
\cellcolor{gray} &  & \verb~3~ & & \verb~namespace Test~\\
\cellcolor{gray} &  & \verb~4~ & & \verb~{~\\
\cellcolor{gray} &  & \verb~5~ & & \verb~    class TestClass~\\
\cellcolor{gray} &  & \verb~6~ & & \verb~    {~\\
\cellcolor{gray} &  & \verb~7~ & & \verb~        public void SampleFunction()~\\
\cellcolor{green} & 1 & \verb~8~ & & \verb~        {~\\
\cellcolor{green} & 1 & \verb~9~ & & \verb~            string test = string.Format(~\\
\cellcolor{green} & 1 & \verb~10~ & & \verb~                "{0} {1}",~\\
\cellcolor{green} & 1 & \verb~11~ & & \verb~                 "Hello",~\\
\cellcolor{green} & 1 & \verb~12~ & & \verb~                 "World");~\\
\cellcolor{gray} &  & \verb~13~ & & \verb~~\\
\cellcolor{green} & 1 & \verb~14~ & & \verb~            Console.WriteLine(test);~\\
\cellcolor{green} & 1 & \verb~15~ & & \verb~            int i = 10;~\\
\cellcolor{gray} &  & \verb~16~ & & \verb~~\\
\cellcolor{orange} & 1 & \verb~17~ & & \verb~            if (i > 0 || i > 1)~\\
\cellcolor{green} & 1 & \verb~18~ & & \verb~            {~\\
\cellcolor{green} & 1 & \verb~19~ & & \verb~                Console.WriteLine(i + " is greater that 0");~\\
\cellcolor{green} & 1 & \verb~20~ & & \verb~            }~\\
\cellcolor{gray} &  & \verb~21~ & & \verb~            else~\\
\cellcolor{red} & 0 & \verb~22~ & & \verb~            {~\\
\cellcolor{red} & 0 & \verb~23~ & & \verb~                Console.WriteLine(i + " is not greater that 0");~\\
\cellcolor{red} & 0 & \verb~24~ & & \verb~            }~\\
\cellcolor{green} & 1 & \verb~25~ & & \verb~        }~\\
\cellcolor{gray} &  & \verb~26~ & & \verb~~\\
\cellcolor{gray} &  & \verb~27~ & & \verb~        public class NestedClass~\\
\cellcolor{gray} &  & \verb~28~ & & \verb~        {~\\
\cellcolor{gray} &  & \verb~29~ & & \verb~            public void SampleFunction()~\\
\cellcolor{red} & 0 & \verb~30~ & & \verb~            {~\\
\cellcolor{red} & 0 & \verb~31~ & & \verb~                Console.WriteLine(~\\
\cellcolor{red} & 0 & \verb~32~ & & \verb~                    "{0} {1}",~\\
\cellcolor{red} & 0 & \verb~33~ & & \verb~                     "Hello",~\\
\cellcolor{red} & 0 & \verb~34~ & & \verb~                     "World");~\\
\cellcolor{red} & 0 & \verb~35~ & & \verb~            }~\\
\cellcolor{gray} &  & \verb~36~ & & \verb~        }~\\
\cellcolor{gray} &  & \verb~37~ & & \verb~    }~\\
\cellcolor{gray} &  & \verb~38~ & & \verb~}~\\
\end{longtable}
\newpage
\section{Test.Program}
\subsection{Summary}
\begin{longtable}[l]{ll}
\textbf{Class:} & Test.Program\\
\textbf{Assembly:} & Sample\\
\textbf{File(s):} & \begin{minipage}[t]{12cm}{C:\textbackslash temp\textbackslash Program.cs}\end{minipage} \\
\textbf{Covered lines:} & 35\\
\textbf{Uncovered lines:} & 9\\
\textbf{Coverable lines:} & 44\\
\textbf{Total lines:} & 84\\
\textbf{Line coverage:} & 79.5\% (35 of 44)\\
\textbf{Covered branches:} & 0\\
\textbf{Total branches:} & 0\\
\textbf{Covered methods:} & 4\\
\textbf{Total methods:} & 6\\
\textbf{Method coverage:} & 66.6\% (4 of 6)\\
\end{longtable}
\subsection{Metrics}
\begin{longtable}[l]{|l|r|r|r|r|r|}
\hline
\textbf{Method} & \textbf{Branch coverage} & \textbf{Crap Score} & \textbf{Cyclomatic complexity} & \textbf{NPath complexity} & \textbf{Sequence coverage}\\
\hline
\textbf{Main(...)} & 100\% & 1.00 & 1 & 0 & 88\%\\
\hline
\textbf{CSharp\_ExecuteTest1(} & 0\% & 2 & 1 & 0 & 0\%\\
\hline
\textbf{CSharp\_ExecuteTest2(} & 0\% & 2 & 1 & 0 & 0\%\\
\hline
\textbf{CallAsyncMethod()} & 100\% & 3 & 3 & 0 & 100\%\\
\hline
\textbf{.ctor(...)} & 100\% & 1 & 1 & 0 & 100\%\\
\hline
\textbf{SendAsync(...)} & 100\% & 1 & 1 & 0 & 100\%\\
\hline
\end{longtable}
\subsection{File(s)}
\subsubsection{C:\textbackslash temp\textbackslash Program.cs}
\begin{longtable}[l]{lrrll}
\textbf{} & \textbf{\#} & \textbf{Line} & \textbf{} & \textbf{Line coverage}\\
\cellcolor{gray} &  & \verb~1~ & & \verb~using System.Net.Http;~\\
\cellcolor{gray} &  & \verb~2~ & & \verb~using System.Threading;~\\
\cellcolor{gray} &  & \verb~3~ & & \verb~using System.Threading.Tasks;~\\
\cellcolor{gray} &  & \verb~4~ & & \verb~using Microsoft.VisualStudio.TestTools.UnitTesting;~\\
\cellcolor{gray} &  & \verb~5~ & & \verb~~\\
\cellcolor{gray} &  & \verb~6~ & & \verb~namespace Test~\\
\cellcolor{gray} &  & \verb~7~ & & \verb~{~\\
\cellcolor{gray} &  & \verb~8~ & & \verb~    [TestClass]~\\
\cellcolor{gray} &  & \verb~9~ & & \verb~    public class Program~\\
\cellcolor{gray} &  & \verb~10~ & & \verb~    {~\\
\cellcolor{gray} &  & \verb~11~ & & \verb~        static void Main(string[] args)~\\
\cellcolor{green} & 1 & \verb~12~ & & \verb~        {~\\
\cellcolor{green} & 1 & \verb~13~ & & \verb~            new TestClass().SampleFunction();~\\
\cellcolor{gray} &  & \verb~14~ & & \verb~~\\
\cellcolor{green} & 1 & \verb~15~ & & \verb~            new TestClass2("Test").ExecutedMethod();~\\
\cellcolor{green} & 1 & \verb~16~ & & \verb~            new TestClass2("Test").SampleFunction("Munich");~\\
\cellcolor{gray} &  & \verb~17~ & & \verb~~\\
\cellcolor{green} & 1 & \verb~18~ & & \verb~            new PartialClass().ExecutedMethod_1();~\\
\cellcolor{green} & 1 & \verb~19~ & & \verb~            new PartialClass().ExecutedMethod_2();~\\
\cellcolor{green} & 1 & \verb~20~ & & \verb~            new PartialClass().SomeProperty = -10;~\\
\cellcolor{gray} &  & \verb~21~ & & \verb~~\\
\cellcolor{green} & 1 & \verb~22~ & & \verb~            new PartialClassWithAutoProperties().Property1 = "Test";~\\
\cellcolor{green} & 1 & \verb~23~ & & \verb~            new PartialClassWithAutoProperties().Property2 = "Test";~\\
\cellcolor{gray} &  & \verb~24~ & & \verb~~\\
\cellcolor{green} & 1 & \verb~25~ & & \verb~            new SomeClass().Property1 = "Test";~\\
\cellcolor{gray} &  & \verb~26~ & & \verb~~\\
\cellcolor{green} & 1 & \verb~27~ & & \verb~            new ClassWithExcludes().IncludedMethod();~\\
\cellcolor{green} & 1 & \verb~28~ & & \verb~            new ClassWithExcludes().ExcludedMethod();~\\
\cellcolor{gray} &  & \verb~29~ & & \verb~~\\
\cellcolor{green} & 1 & \verb~30~ & & \verb~            new GenericClass<SomeModel, IState>().Process(null);~\\
\cellcolor{green} & 1 & \verb~31~ & & \verb~            new GenericClass<SomeModel, IState>().PostProcess(null);~\\
\cellcolor{gray} &  & \verb~32~ & & \verb~~\\
\cellcolor{green} & 1 & \verb~33~ & & \verb~            new CodeContract_Target().Calculate(-1);~\\
\cellcolor{gray} &  & \verb~34~ & & \verb~~\\
\cellcolor{green} & 1 & \verb~35~ & & \verb~            new AbstractClass_SampleImpl1();~\\
\cellcolor{green} & 1 & \verb~36~ & & \verb~            new AbstractClass_SampleImpl2();~\\
\cellcolor{gray} &  & \verb~37~ & & \verb~~\\
\cellcolor{green} & 1 & \verb~38~ & & \verb~            CallAsyncMethod();~\\
\cellcolor{gray} &  & \verb~39~ & & \verb~~\\
\cellcolor{gray} &  & \verb~40~ & & \verb~            try~\\
\cellcolor{green} & 1 & \verb~41~ & & \verb~            {~\\
\cellcolor{green} & 1 & \verb~42~ & & \verb~                new CodeContract_Target().Calculate(0);~\\
\cellcolor{green} & 1 & \verb~43~ & & \verb~            }~\\
\cellcolor{red} & 0 & \verb~44~ & & \verb~            catch (System.ArgumentException)~\\
\cellcolor{red} & 0 & \verb~45~ & & \verb~            {~\\
\cellcolor{red} & 0 & \verb~46~ & & \verb~            }~\\
\cellcolor{green} & 1 & \verb~47~ & & \verb~        }~\\
\cellcolor{gray} &  & \verb~48~ & & \verb~~\\
\cellcolor{gray} &  & \verb~49~ & & \verb~        [TestMethod]~\\
\cellcolor{gray} &  & \verb~50~ & & \verb~        public void CSharp_ExecuteTest1()~\\
\cellcolor{red} & 0 & \verb~51~ & & \verb~        {~\\
\cellcolor{red} & 0 & \verb~52~ & & \verb~            Main(null);~\\
\cellcolor{red} & 0 & \verb~53~ & & \verb~        }~\\
\cellcolor{gray} &  & \verb~54~ & & \verb~~\\
\cellcolor{gray} &  & \verb~55~ & & \verb~        [TestMethod]~\\
\cellcolor{gray} &  & \verb~56~ & & \verb~        public void CSharp_ExecuteTest2()~\\
\cellcolor{red} & 0 & \verb~57~ & & \verb~        {~\\
\cellcolor{red} & 0 & \verb~58~ & & \verb~            Main(null);~\\
\cellcolor{red} & 0 & \verb~59~ & & \verb~        }~\\
\cellcolor{gray} &  & \verb~60~ & & \verb~~\\
\cellcolor{gray} &  & \verb~61~ & & \verb~        private static async void CallAsyncMethod()~\\
\cellcolor{green} & 1 & \verb~62~ & & \verb~        {~\\
\cellcolor{green} & 1 & \verb~63~ & & \verb~            var expected = new HttpResponseMessage();~\\
\cellcolor{green} & 1 & \verb~64~ & & \verb~            var handler = new AsyncClass() { InnerHandler = new EchoHandler(expected) };~\\
\cellcolor{green} & 1 & \verb~65~ & & \verb~            var invoker = new HttpMessageInvoker(handler, false);~\\
\cellcolor{green} & 1 & \verb~66~ & & \verb~            var actual = await invoker.SendAsync(new HttpRequestMessage(), new CancellationToken());~\\
\cellcolor{green} & 1 & \verb~67~ & & \verb~        }~\\
\cellcolor{gray} &  & \verb~68~ & & \verb~~\\
\cellcolor{gray} &  & \verb~69~ & & \verb~        private class EchoHandler : DelegatingHandler~\\
\cellcolor{gray} &  & \verb~70~ & & \verb~        {~\\
\cellcolor{gray} &  & \verb~71~ & & \verb~            private HttpResponseMessage _response;~\\
\cellcolor{gray} &  & \verb~72~ & & \verb~~\\
\cellcolor{green} & 1 & \verb~73~ & & \verb~            public EchoHandler(HttpResponseMessage response)~\\
\cellcolor{green} & 1 & \verb~74~ & & \verb~            {~\\
\cellcolor{green} & 1 & \verb~75~ & & \verb~                this._response = response;~\\
\cellcolor{green} & 1 & \verb~76~ & & \verb~            }~\\
\cellcolor{gray} &  & \verb~77~ & & \verb~~\\
\cellcolor{gray} &  & \verb~78~ & & \verb~            protected override Task<HttpResponseMessage> SendAsync(HttpRequestMessage request, CancellationToken cancell~\\
\cellcolor{green} & 1 & \verb~79~ & & \verb~            {~\\
\cellcolor{green} & 1 & \verb~80~ & & \verb~                return Task.FromResult(this._response);~\\
\cellcolor{green} & 1 & \verb~81~ & & \verb~            }~\\
\cellcolor{gray} &  & \verb~82~ & & \verb~        }~\\
\cellcolor{gray} &  & \verb~83~ & & \verb~    }~\\
\cellcolor{gray} &  & \verb~84~ & & \verb~}~\\
\end{longtable}
\newpage
\section{Test.TestClass2}
\subsection{Summary}
\begin{longtable}[l]{ll}
\textbf{Class:} & Test.TestClass2\\
\textbf{Assembly:} & Sample\\
\textbf{File(s):} & \begin{minipage}[t]{12cm}{C:\textbackslash temp\textbackslash TestClass2.cs}\end{minipage} \\
\textbf{Covered lines:} & 24\\
\textbf{Uncovered lines:} & 14\\
\textbf{Coverable lines:} & 38\\
\textbf{Total lines:} & 85\\
\textbf{Line coverage:} & 63.1\% (24 of 38)\\
\textbf{Covered branches:} & 1\\
\textbf{Total branches:} & 2\\
\textbf{Branch coverage:} & 50\% (1 of 2)\\
\textbf{Covered methods:} & 6\\
\textbf{Total methods:} & 10\\
\textbf{Method coverage:} & 60\% (6 of 10)\\
\end{longtable}
\subsection{Metrics}
\begin{longtable}[l]{|l|r|r|r|r|r|}
\hline
\textbf{Method} & \textbf{Branch coverage} & \textbf{Crap Score} & \textbf{Cyclomatic complexity} & \textbf{NPath complexity} & \textbf{Sequence coverage}\\
\hline
\textbf{.ctor()} & 0\% & 2 & 1 & 0 & 0\%\\
\hline
\textbf{.ctor(...)} & 100\% & 1 & 1 & 0 & 100\%\\
\hline
\textbf{ExecutedMethod()} & 100\% & 1 & 1 & 0 & 100\%\\
\hline
\textbf{UnExecutedMethod()} & 0\% & 2 & 1 & 0 & 0\%\\
\hline
\textbf{SampleFunction(...)} & 66.67\% & 5 & 5 & 2 & 100\%\\
\hline
\textbf{DoSomething(...)} & 0\% & 2 & 1 & 0 & 0\%\\
\hline
\end{longtable}
\subsection{File(s)}
\subsubsection{C:\textbackslash temp\textbackslash TestClass2.cs}
\begin{longtable}[l]{lrrll}
\textbf{} & \textbf{\#} & \textbf{Line} & \textbf{} & \textbf{Line coverage}\\
\cellcolor{gray} &  & \verb~1~ & & \verb~using System;~\\
\cellcolor{gray} &  & \verb~2~ & & \verb~using System.Collections.Generic;~\\
\cellcolor{gray} &  & \verb~3~ & & \verb~using System.Linq;~\\
\cellcolor{gray} &  & \verb~4~ & & \verb~~\\
\cellcolor{gray} &  & \verb~5~ & & \verb~namespace Test~\\
\cellcolor{gray} &  & \verb~6~ & & \verb~{~\\
\cellcolor{gray} &  & \verb~7~ & & \verb~    class TestClass2~\\
\cellcolor{gray} &  & \verb~8~ & & \verb~    {~\\
\cellcolor{gray} &  & \verb~9~ & & \verb~        private string name;~\\
\cellcolor{gray} &  & \verb~10~ & & \verb~~\\
\cellcolor{green} & 2 & \verb~11~ & & \verb~        private Dictionary<string, int> dict = new Dictionary<string, int>();~\\
\cellcolor{gray} &  & \verb~12~ & & \verb~~\\
\cellcolor{green} & 3 & \verb~13~ & & \verb~        public string ExecutedProperty { get; set; }~\\
\cellcolor{gray} &  & \verb~14~ & & \verb~~\\
\cellcolor{red} & 0 & \verb~15~ & & \verb~        public string UnExecutedProperty { get; set; }~\\
\cellcolor{gray} &  & \verb~16~ & & \verb~~\\
\cellcolor{red} & 0 & \verb~17~ & & \verb~        public TestClass2()~\\
\cellcolor{red} & 0 & \verb~18~ & & \verb~        {~\\
\cellcolor{red} & 0 & \verb~19~ & & \verb~            this.name = "Nobody";~\\
\cellcolor{red} & 0 & \verb~20~ & & \verb~            this.ExecutedProperty = "Nobody";~\\
\cellcolor{red} & 0 & \verb~21~ & & \verb~        }~\\
\cellcolor{gray} &  & \verb~22~ & & \verb~~\\
\cellcolor{green} & 2 & \verb~23~ & & \verb~        public TestClass2(string name)~\\
\cellcolor{green} & 2 & \verb~24~ & & \verb~        {~\\
\cellcolor{green} & 2 & \verb~25~ & & \verb~            this.name = name;~\\
\cellcolor{green} & 2 & \verb~26~ & & \verb~            this.ExecutedProperty = name + name;~\\
\cellcolor{green} & 2 & \verb~27~ & & \verb~        }~\\
\cellcolor{gray} &  & \verb~28~ & & \verb~~\\
\cellcolor{gray} &  & \verb~29~ & & \verb~        public void ExecutedMethod()~\\
\cellcolor{green} & 1 & \verb~30~ & & \verb~        {~\\
\cellcolor{green} & 1 & \verb~31~ & & \verb~            Console.WriteLine(this.name);~\\
\cellcolor{green} & 1 & \verb~32~ & & \verb~            Console.WriteLine(this.ExecutedProperty);~\\
\cellcolor{green} & 1 & \verb~33~ & & \verb~        }~\\
\cellcolor{gray} &  & \verb~34~ & & \verb~~\\
\cellcolor{gray} &  & \verb~35~ & & \verb~        public void UnExecutedMethod()~\\
\cellcolor{red} & 0 & \verb~36~ & & \verb~        {~\\
\cellcolor{red} & 0 & \verb~37~ & & \verb~            Console.WriteLine(this.name);~\\
\cellcolor{red} & 0 & \verb~38~ & & \verb~            Console.WriteLine(this.ExecutedProperty);~\\
\cellcolor{red} & 0 & \verb~39~ & & \verb~        }~\\
\cellcolor{gray} &  & \verb~40~ & & \verb~~\\
\cellcolor{gray} &  & \verb~41~ & & \verb~        public void SampleFunction(string city)~\\
\cellcolor{green} & 1 & \verb~42~ & & \verb~        {~\\
\cellcolor{green} & 1 & \verb~43~ & & \verb~            int[] values = new int[] { 0, 1, 2, 3 };~\\
\cellcolor{gray} &  & \verb~44~ & & \verb~~\\
\cellcolor{green} & 5 & \verb~45~ & & \verb~            var doubled = values.Select(i => i * 2);~\\
\cellcolor{gray} &  & \verb~46~ & & \verb~~\\
\cellcolor{green} & 11 & \verb~47~ & & \verb~            foreach (var item in doubled)~\\
\cellcolor{green} & 4 & \verb~48~ & & \verb~            {~\\
\cellcolor{green} & 4 & \verb~49~ & & \verb~                Console.WriteLine(item);~\\
\cellcolor{green} & 4 & \verb~50~ & & \verb~            }~\\
\cellcolor{gray} &  & \verb~51~ & & \verb~~\\
\cellcolor{green} & 1 & \verb~52~ & & \verb~            string[] cities = new string[] { "Berlin", "Munich", "Paris" };~\\
\cellcolor{gray} &  & \verb~53~ & & \verb~~\\
\cellcolor{orange} & 4 & \verb~54~ & & \verb~            if (cities.SingleOrDefault(c => c.Equals(city, StringComparison.OrdinalIgnoreCase)) != null)~\\
\cellcolor{green} & 1 & \verb~55~ & & \verb~            {~\\
\cellcolor{green} & 1 & \verb~56~ & & \verb~                Console.WriteLine("Found " + city);~\\
\cellcolor{green} & 1 & \verb~57~ & & \verb~            }~\\
\cellcolor{green} & 1 & \verb~58~ & & \verb~        }~\\
\cellcolor{gray} &  & \verb~59~ & & \verb~~\\
\cellcolor{gray} &  & \verb~60~ & & \verb~        public string DoSomething(string value,~\\
\cellcolor{gray} &  & \verb~61~ & & \verb~            string[] stringArray,~\\
\cellcolor{gray} &  & \verb~62~ & & \verb~            Guid id,~\\
\cellcolor{gray} &  & \verb~63~ & & \verb~            IEnumerable<string> stringEnumerable,~\\
\cellcolor{gray} &  & \verb~64~ & & \verb~            IList<string> stringList,~\\
\cellcolor{gray} &  & \verb~65~ & & \verb~            decimal dec,~\\
\cellcolor{gray} &  & \verb~66~ & & \verb~            int i,~\\
\cellcolor{gray} &  & \verb~67~ & & \verb~            Dictionary<string, int> dict,~\\
\cellcolor{gray} &  & \verb~68~ & & \verb~            out int g,~\\
\cellcolor{gray} &  & \verb~69~ & & \verb~            float fff,~\\
\cellcolor{gray} &  & \verb~70~ & & \verb~            double dou,~\\
\cellcolor{gray} &  & \verb~71~ & & \verb~            bool bo,~\\
\cellcolor{gray} &  & \verb~72~ & & \verb~            byte by,~\\
\cellcolor{gray} &  & \verb~73~ & & \verb~            char ch,~\\
\cellcolor{gray} &  & \verb~74~ & & \verb~            object o,~\\
\cellcolor{gray} &  & \verb~75~ & & \verb~            sbyte sby,~\\
\cellcolor{gray} &  & \verb~76~ & & \verb~            short sh,~\\
\cellcolor{gray} &  & \verb~77~ & & \verb~            uint ui,~\\
\cellcolor{gray} &  & \verb~78~ & & \verb~            ulong ul,~\\
\cellcolor{gray} &  & \verb~79~ & & \verb~            ushort usho)~\\
\cellcolor{red} & 0 & \verb~80~ & & \verb~        {~\\
\cellcolor{red} & 0 & \verb~81~ & & \verb~            g = 0;~\\
\cellcolor{red} & 0 & \verb~82~ & & \verb~            return null;~\\
\cellcolor{red} & 0 & \verb~83~ & & \verb~        }~\\
\cellcolor{gray} &  & \verb~84~ & & \verb~    }~\\
\cellcolor{gray} &  & \verb~85~ & & \verb~}~\\
\end{longtable}
\end{document}